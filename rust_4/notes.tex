% Skip over this boring header down a bit
%%%%%%%%%%%%%%%%%%%%%%%%%%%%%%%%%%%%%%%%%%%%%%%%
\documentclass[usenames,twocolumn,dvipsnames,10pt,a4wide]{article} 
\usepackage[a4paper, total={8in, 10in}]{geometry}
% Add option 'aspectratio=169' for 16:9 widescreen 
% Add option  'handout' to ignore animations
% If you have a smaller amount of text, feel free to also try '11pt'! / Jesper

\usepackage[utf8]{inputenc}
\usepackage{verbatim}
\usepackage{minted}
\usemintedstyle{vs}
\usepackage{graphicx}
\usepackage{wrapfig}
\usepackage[document]{ragged2e}

\usepackage[shortlabels]{enumitem}

%%% Bibliography
\usepackage[style=authoryear,backend=biber]{biblatex}
\addbibresource{bibliography.bib}

\DeclareNameAlias{author}{given-family}

%%% Suppress biblatex annoying warning
\usepackage{silence}
\WarningFilter{biblatex}{Patching footnotes failed}

%%% Some useful commands
% pdf-friendly newline in links
\newcommand{\pdfnewline}{\texorpdfstring{\newline}{ }} 
% Fill the vertical space in a slide (to put text at the bottom)
\newcommand{\framefill}{\vskip0pt plus 1filll}

%%%%%%%%%%%%%%%%%%%%%%%%%%%%%%%%%%%%%%%%%%%%%%%%%%%%%%%%%%%%%%%%%%%%%%%%%%%%%%%%%%%%%
\title{Rust \#1: Motivation and Introduction}

\begin{document}
\maketitle


\section{What are we doing here?}

\subsection{Why are we here?}
We are here to learn a new way 
of thinking about complex computer 
systems with the help of the Rust 
programming language.


\subsection{What this entails}
We will have to do a few things in order to gain
the most in the short amount of time we have:

\begin{itemize}[label=$\bullet$]
	\item Listen and ask questions during the lecture
	\item Complete short homework exercises
	\item Work our way through a final group project
	\item Read/Watch/Listen/Do anything and everything 
		we can lay our eyes upon!
\end{itemize}


\subsection{Rust doesn't solve all of your problems}
While Rust does solve most memory safety issues
(more on this later), most bugs are still caused
by tired and underslept programmers.

Your health is more important than your work.


\subsection{Rust doesn't solve all of your problems}
Developing more secure and memory-safe code in
abstract sounds good. But it's not always.


Developing a secure drone strike system won't help anyone.


Secure face recognition system for the police won't help anyone either.


A secure app that serves as a tool to further underpay and
undermine the vulnerable workforce will bring far more trouble
if it's unhackable using simple memory exploits.



\section{Motivation}%
 
\subsection{Who is interested in Rust?}
	%Leaving all the technical points for later	
	Rust is the most loved language for the fifth year in a row!
	\footnotesize{(86.\% of StackOverflow users)}

	

	Rust is used in production in an increasing number
	of companies: Mozilla, Atlassian, Microsoft, Google,
	Godot, 1Password, Dropbox, Zeplin, npm, Academia.edu,
	Sentry, Cloudflare, Coursera, Figma, Postmates etc.

	

	But Rust's concepts will help in many
	other languages as well!
 

\subsection{Why should you be interested in Rust?}
	Rust is an extremely interesting and relatively
	new language that can be used in a lot of
	domains, including, but not limited to:
	
	\begin{itemize}[label=$\bullet$]
		\item Web through WebAssembly
		\item Operating Systems
		\item Embedded Software
		\item Large Systems Software
			(Browsers, Servers, Databases etc.)
		\item Games, GUI, CLI applications too!
	\end{itemize}


\subsection{Resources}
	All of these can be found in our repository!
	And even more are available online!
	





\section{A short history of systems programming}

\subsection{The origins of C}
\normalsize
The C programming language appeared during Unix development in 1972.

Since it was created for a specific purpose % an operating system
and a specific computer, %(PDP-11) 
on the one hand it adapted to the
needs of the programmers, and on the 
other it adopted a large
amount of somewhat unique and unpopular 
ideas and concepts.

\vskip 0.8cm



\subsection{Possible solutions}
C++ is born to help address some of these problems, 
introduces ‘zero cost’ abstractions, aimed at providing
a nice interface for the programmer to use which
compiles down to an almost ideal machine code.


Still has the old instruments, hangs on to C's machine
model %(the fucking BACKWARDS COMPATIBILITY),
and tries to encourage using the new 
modern safe concepts, 
which are not ideal either




\subsection{Modern ideas}

In the meantime, languages like Java, Ruby and
Python start sprawling up, presenting another
model of growth - they are garbage-collected
and are able to present even more complex
abstractions (at the expense of the speed).



Go and others try to tackle C's speed and
low-levelness, unsuccessfully.



\section{Pitfalls of the old ways}

\subsection{Memory layout} 
\inputminted[fontsize=\normalsize]{c}{code/stack.c}


\subsection{Memory layout} 
stack picture
	

\subsection{Buffer overflows} 
	\inputminted[fontsize=\normalsize]{c}{code/overflow.c}


\subsection{Null pointers}
	\inputminted[fontsize=\normalsize]{c}{code/nullp.c}


\subsection{Dangling pointers} 
	\inputminted[fontsize=\normalsize]{c}{code/danp.c}


\subsection{Invalidated iterators and double free} 
	\inputminted[fontsize=\normalsize]{c}{code/iter.c}


\subsection{No real error checking} 
	\inputminted[fontsize=\normalsize]{c}{code/errorcheck.c}


\subsection{And many more...}
	\begin{itemize}[label=$\bullet$]
		\item Memory leak - you can forget to free data
		\item Thread unsafety - another function can be 
			modifying the same memory
	\end{itemize}




\section{Where and why is Rust better?}

\subsection{What is Rust?}



A modern systems programming language.



First stable version in 2014.



Gets rid of unnecessary old ideas, combining 
them with some of the fresh concepts.


\subsection{Almost everything makes sense} 

It does not care about C’s old ways 
from the 70s which have been kept up 
in many languages and systems since. 
%just because (null-char strings, pointers) 


It does not try to needlessly attach new 
stupid things to it. Easily gets rid of bad 
ideas since it’s a young language.


\subsection{Memory safety guarantees}
Rust guarantees, at compile time, that
programs are memory-safe.



This means:
\begin{itemize}[label=$\bullet$]
	\item No buffer overflows
	\item Everything is bounds-checked (e.g. no null pointers, no dangling pointers)
	\item No data races (general thread-safety)
	\item No memory is ever written to by two things at a time
	\item No memory is ever written and read at the same time
\end{itemize}


\subsection{Tradeoffs}
These guarantees don't
come without a cost.



Some of them include:
\begin{itemize}[label=$\bullet$]
	\item Relatively long compile time
	\item You sometimes have to fight with the compiler
		(It's always right though, and even tries to help)
	\item A whole new different approach to things
	\item It's a young language!
		(This also means you can help)
\end{itemize}


\subsection{Improvements in almost every field} 
Rust does not only get rid of the problems
of C and garbage-collected languages.


It tries to be better at a lot of other things:
\begin{itemize}[label=$\bullet$]
	\item Super great error handling!
	\item Algebraic data types!
		(Functional programming concept)
	\item Advanced macro processing!
	\item Speed increase from C since programs 
		often know what to expect!
\end{itemize}


\subsection{An amazing ecosystem} 
%Modern languages are not only language
%specifications. Their use almost always
%depends on compilers, package management
%systems, documentation, formatters, and
%test suites!


Rust has a great ecosystem:
\begin{itemize}[label=$\bullet$]
	\item rustc compiler is super helpful, and has great IDE integrations!
	\item cargo is a unified standard for:
		\begin{itemize}[label=$\bullet$]
			\item Package management (pip, yay)
			\item Dependency resolving (setup.py, requirements.txt)
			\item Project management (makefile, CMake)
			\item Testing (gtest etc.)
			\item Documentation (pandoc, asciidoc etc.)
		\end{itemize}
	\item rustfmt is a standard formatter
	\item clippy is a standard linter
\end{itemize}




\section{Basic syntax and ecosystem}

\subsection{Hello world!}
	All of the code snippets for this lecture
	are available in the repository!
	
	\inputminted[fontsize=\normalsize]{rust}{code/helloworld.rs}
	
	\inputminted[fontsize=\normalsize]{bash}{code/helloworld.sh}



\subsection{Hello cargo!}
	\inputminted[fontsize=\normalsize]{bash}{code/hellocargo.sh}



\section{Ownership and lifetimes}
\subsection{Ownership and lifetimes}
	\subsubsection{Abstract}
	How would a safe system look like?
	
	Let's go over a few main points:
	
	\begin{itemize}[label=$\bullet$]
		\item There can be many readers with no writers at the same time
		\item There can be only one writer with no readers at the same time
		\item Values can be used only as long as they still exist
	\end{itemize}


\subsection{Ownership and lifetimes}
	\subsubsection{General}
	In Rust's terminology, there are two notions of borrowing:
	\begin{itemize}[label=$\bullet$]
		\item Shared reference \&
		\item Mutable reference \&mut
	\end{itemize}

	
	Rust forces us to abide by the following principles:
	\begin{itemize}[label=$\bullet$]
		\item Every value has a single owner at any given time.
			(You can move a value from one owner to another, but when
			value's owner goes away, the value goes away with it too)
		\item You can borrow a reference to a value, for as long as
			the reference doesn't outlive the value.
			(Borrowed references are temporary pointers, they
			allow you to operate with values you don't own)
		\item You can only modify a value when you have exclusive
			access to it.
	\end{itemize}
	


\subsection{Ownership and lifetimes}
	\subsubsection{General}
	The lifetime of a value starts when it’s created
	and ends the last time it’s used.
	
	Rust computes lifetimes at compile time, we don't
	have to do any allocations and frees, Rust borrows
	and drops memory for us.


\subsection{Ownership and lifetimes}
	\subsubsection{Examples}
	Let's see how Rust automatically drops any values after
	their scope ends (and their owner dies):
	
	\inputminted[fontsize=\normalsize]{rust}{code/own1.rs}
	
	Rust handles all data this way, no matter the complexity,
	it will clean up file handlers, network sockets, vectors etc.


\subsection{Ownership and lifetimes}
	\subsubsection{Examples}
	Let's see how Rust handles Strings. Does this compile?
	
	\inputminted[fontsize=\normalsize]{rust}{code/own2.rs}


\subsection{Ownership and lifetimes}
	\subsubsection{Examples}
	Does this one compile?
	
	\inputminted[fontsize=\normalsize]{rust}{code/own3.rs}


\subsection{Ownership and lifetimes}
	\subsubsection{Examples}
	What is different for u32?
	
	\inputminted[fontsize=\normalsize]{rust}{code/own4.rs}


\subsection{Ownership and lifetimes}
	\subsubsection{Copy and Move}
	Rust has a special notion which helps it figure
	out whether the value should be copied or moved.
	
	We'll discuss this a bit later, but for now you	can
	just remember that Rust copies stack-allocated
	primitive values, and moves ownership of heap-
	allocated values.
	


\subsection{Ownership and lifetimes}
	\subsubsection{Examples}
	\inputminted[fontsize=\normalsize]{rust}{code/own5.rs}


\subsection{Ownership and lifetimes}
	\subsubsection{Examples}
	\inputminted[fontsize=\normalsize]{rust}{code/own6.rs}


\subsection{Ownership and lifetimes}
	\subsubsection{Examples}
	Let's see how Rust will handle complex correct code:
	
	\inputminted[fontsize=\normalsize]{rust}{code/own7.rs}
	


\subsection{Ownership and lifetimes}
	\subsubsection{Examples}
	Let's try to invalidate an iterator:
	
	\inputminted[fontsize=\normalsize]{rust}{code/own8.rs}
	


\subsection{Ownership and lifetimes}
	\subsubsection{Examples}
	Let's try to return a dangling pointer:
	
	\inputminted[fontsize=\normalsize]{rust}{code/own9.rs}
	


\subsection{Ownership and lifetimes}
We've only looked at some of the simplest cases,
Rust's system is a lot more complex. We'll go
over lifetimes more in the future.

Since the concept of borrow checker is unique to
Rust, until you will get used to it you will often
encounter compiler errors when working.

The compiler becomes better and better at working
with lifetimes, and provides super helpful errors,
but it still can be stressful to fight with it
sometimes.


\section{A simple practice project}

\subsection{Guessing game}
	\subsubsection{Basics}
	\inputminted[fontsize=\footnotesize]{rust}{code/guess1.rs}
	


\subsection{Guessing game}
	\subsubsection{Cargo dependencies}
	Let's add a random number generator. We have
	to add a dependency for rand crate for this.
	(crates are Rust's libraries and modules)
	

	Add this to your Cargo.toml file:
	
	\inputminted[fontsize=\normalsize]{toml}{code/toml1.toml}


\subsection{Guessing game}
	\subsubsection{Random number generation}
	\inputminted[fontsize=\footnotesize]{rust}{code/guess2.rs}


\subsection{Guessing game}
	\subsubsection{(Wrong) Input comparison}
	%Match expression, enumerators explanation.*
	\inputminted[fontsize=\normalsize]{rust}{code/guess3.rs}


\subsection{Guessing game}
	\subsubsection{Input comparison}
	%Ownership explanation, maybe add a slide about that?*
	\inputminted[fontsize=\normalsize]{rust}{code/guess4.rs}



\subsection{Guessing game}
	\subsubsection{Looping}
	\inputminted[fontsize=\normalsize]{rust}{code/guess5.rs}


\subsection{Guessing game}
	\subsubsection{Breaking out of the loop}
	\inputminted[fontsize=\normalsize]{rust}{code/guess6.rs}


\subsection{Guessing game}
	\subsubsection{Handling invalid input}
	\inputminted[fontsize=\normalsize]{rust}{code/guess7.rs}


\subsection{More on cargo}
	\inputminted[fontsize=\normalsize]{shell}{code/cargo.sh}

\end{document}
