\documentclass[usenames,dvipsnames,10pt,aspectratio=169]{beamer} 
% Add option 'aspectratio=169' for 16:9 widescreen 
% Add option  'handout' to ignore animations
% If you have a smaller amount of text, feel free to also try '11pt'! / Jesper

\usepackage[utf8]{inputenc}
\usepackage{verbatim}
\usepackage{minted}
\usemintedstyle{monokai}
\usepackage{graphicx}
\usepackage{wrapfig}
\usepackage[document]{ragged2e}
\usetheme{umu}

\usepackage{hyperref}
\hypersetup{
    colorlinks=true,
    linkcolor=ucuyellow,
    filecolor=ucuyellow,      
    urlcolor=ucuyellow,
}
\urlstyle{same}

%%% Bibliography
\usepackage[style=authoryear,backend=biber]{biblatex}
\addbibresource{bibliography.bib}

\DeclareNameAlias{author}{given-family}

%%% Suppress biblatex annoying warning
\usepackage{silence}
\WarningFilter{biblatex}{Patching footnotes failed}

%%% Some useful commands
% pdf-friendly newline in links
\newcommand{\pdfnewline}{\texorpdfstring{\newline}{ }} 
% Fill the vertical space in a slide (to put text at the bottom)
\newcommand{\framefill}{\vskip0pt plus 1filll}

%%% Enter additional packages below (or above, I can't stop you)! / Jesper
\renewcommand{\proofname}{\sffamily{Proof}}

%%%%%%%%%%%%%%%%%%%%%%%%%%%%%%%%%%%%%%%%%%%%%%%%%%%%%%%%%%%%%%%%%%%%%%%%%%%%%%%%%%%%%
\title[Rust \#2]{Rust \#2: Ownership,\\ Structs and OO}
\date[\today]{\small\today}
\author[Sultanov Andriy]{Sultanov Andriy}
\institute{APPS@UCU}

\begin{document}

\begin{frame}
\titlepage
\end{frame}

\begin{frame}{\contentsname}
\tableofcontents
\end{frame}

\framepic{graphics/2.jpg}{
 \textcolor{ucuwhite}{Rust basics}
 \vskip 0.5cm
 }

\section{Rust basics}
\begin{frame}{Rust primitive types} 
\centering
\large
Integer types
\begin{table}[]
\begin{tabular}{lll}
Length  & Signed & Unsigned \\
8-bit   & i8     & u8       \\
16-bit  & i16    & u16      \\
32-bit  & i32    & u32      \\
64-bit  & i64    & u64      \\
128-bit & i128   & u128     \\
arch    & isize  & usize   
\end{tabular}
\end{table}
\vspace{0.5cm}
There are also two floating point types:
\textcolor{ucuyellow}{f32} and \textcolor{ucuyellow}{f64}.\\
And \textcolor{ucuyellow}{bool}, \textcolor{ucuyellow}{char} types.
Characters in Rust are Unicode chars, and can take up to 4 bytes.
\end{frame}

\begin{frame}{Rust compound types}
\framesubtitle{Tuples}
Tuple groups together a number of values with different\\
types into one compound type. Tuples have a fixed length.\\
\vspace{0.2cm}
\inputminted[fontsize=\large]{rust}{code/tuple.rs}
\vspace{0.5cm}
\end{frame}

\begin{frame}{Rust compound types}
\framesubtitle{Arrays}
Arrays are a collection of elements of the same type,\\
with a fixed length, allocated on the stack.
\vspace{0.2cm}
\inputminted[fontsize=\large]{rust}{code/array.rs}
\vspace{0.5cm}
\end{frame}

\begin{frame}{Rust compound types}
\framesubtitle{Slices}
Slices are dynamically-sized "windows", "slices" into a\\
collection of elements. Slices let the code that handles\\
them not care whether it's currently working with an\\
array slice, with a vector slice or something else.
\vspace{0.2cm}
\inputminted[fontsize=\large]{rust}{code/slice.rs}
\vspace{0.5cm}
\end{frame}

\begin{frame}{Rust compound types}
\framesubtitle{String slices}
String slices are similar, but instead function as windows\\
into strings. Making your function take a \textcolor{ucuyellow}{str} instead of \\
\textcolor{ucuyellow}{String} is preferrable, since the latter can be\\
downreferenced to slices.
\vspace{0.2cm}
\inputminted[fontsize=\large]{rust}{code/str.rs}
\vspace{0.5cm}
\end{frame}

\begin{frame}{Functions}
\large
An example of a function with parameters and a return type:
\vspace{0.2cm}
\inputminted[fontsize=\large]{rust}{code/function.rs}
\vspace{0.5cm}
\end{frame}

\begin{frame}{Control flow}
	\framesubtitle{Unconditional looping}
\inputminted[fontsize=\Large]{rust}{code/control1.rs}
\vspace{0.7cm}
\end{frame}

\begin{frame}{Control flow}
	\framesubtitle{Conditional looping}
\inputminted[fontsize=\Large]{rust}{code/control2.rs}
\vspace{0.7cm}
\end{frame}

\begin{frame}{Control flow}
	\framesubtitle{Label break}
\inputminted[fontsize=\Large]{rust}{code/control3.rs}
\end{frame}

\framepic{graphics/2.jpg}{
 \textcolor{ucuwhite}{Rust principles}
 \vskip 0.5cm
 }

\section{Rust principles}
\begin{frame}{Rust principles}
\framesubtitle{Expressions and statements}
\large{\textcolor{ucuyellow}{Rust is primarily an expression language.\\}}
\vspace{0.5cm}
Essentially: Expressions evaluate to a value, and\\
return that value. Statements do not.
\vspace{0.5cm}
\inputminted[fontsize=\large]{rust}{code/expression1.rs}
\end{frame}

\begin{frame}{Rust principles}
\framesubtitle{Expressions and statements}
\normalsize
Function bodies are made up of a series of statements, \\
optionally ending in an expression.

\vspace{0.4cm}
Expressions do not include ending semicolons.\\

\vspace{0.4cm}
If you add a semicolon to the end of an expression, you turn \\
it into a statement, which will then not return a value.\\
If a function ends in an expression, it returns the value of\\
that expression.
\vspace{0.5cm}
\inputminted[fontsize=\large]{rust}{code/expression2.rs}
\vspace{0.5cm}
\end{frame}

\begin{frame}{Rust principles}
\framesubtitle{Common expression usage}
\large
Scopes can return values:\\
(Rust returns \textcolor{ucuyellow}{( )} if nothing is returned, it's like None)
\inputminted[fontsize=\large]{rust}{code/expression3.rs}
\vspace{0.7cm}
\textcolor{ucuyellow}{if}
is also an expression:
\inputminted[fontsize=\large]{rust}{code/expression4.rs}
\vspace{0.5cm}
\normalsize
We can return values from a lot of expressions in Rust 
(\textcolor{ucuyellow}{match}, for example)
\end{frame}

\begin{frame}{Rust principles}
\framesubtitle{Algebraic data types and match expressions}
\large
Rust uses an interesting concept of algebraic \\
data types, which can hold a few types of values.\\ 
An example of this is \textcolor{ucuyellow}{std::Option}:\\
\vspace{0.2cm}
\inputminted[fontsize=\Large]{rust}{code/option1.rs}
\vspace{0.4cm}
\end{frame}

\begin{frame}{Rust principles}
\framesubtitle{Algebraic data types and match expressions}
\large
\inputminted[fontsize=\large]{rust}{code/option1.rs}
\vspace{0.4cm}
\normalsize
Rust's \textcolor{ucuyellow}{enum}s are different from C's since they can hold additional\\
values inside of them, and Rust's system makes it so you can't\\
treat one of the values as if it was a different value.\\

\vspace{0.2cm}
An \textcolor{ucuyellow}{Option<T>}
contains either a \textcolor{ucuyellow}{Some(value of type T)}
or \textcolor{ucuyellow}{None}.\\
(These are called enum's variants)\\

Thus, an \textcolor{ucuyellow}{Option<f64>}
is either a \textcolor{ucuyellow}{Some(f64)}
or \textcolor{ucuyellow}{None}.
\end{frame}
\begin{frame}{Rust principles}
\framesubtitle{Algebraic data types and match expressions}
\large	
Let's see how these types can be used:
\vspace{0.2cm}
\inputminted[fontsize=\large]{rust}{code/option2.rs}
\end{frame}

\begin{frame}{Rust principles}
\framesubtitle{Algebraic data types and match expressions}
\large
Rust forces us to consider all the possible values\\
of algebraic data types:
\vspace{0.3cm}
\inputminted[fontsize=\large]{rust}{code/option3.rs}
\vspace{0.3cm}
You can never miss an error or have an unexpected value this way!
\vspace{0.4cm}
\end{frame}

\begin{frame}{Rust principles}
\framesubtitle{Option matching shortcuts}
\large
We can use \textcolor{ucuyellow}{if let} if we only care about the successful\\
case (think of it as a pattern we match against). You\\
can also add an else clause, of course:

\vspace{0.3cm}
\inputminted[fontsize=\Large]{rust}{code/option4.rs}
\end{frame}

\begin{frame}{Rust principles}
\framesubtitle{Option matching shortcuts}
\large
This destructuring can also be used to loop over things,\\
(Rust iterators produce Options, for example):\\

\vspace{0.3cm}
\inputminted[fontsize=\Large]{rust}{code/option5.rs}
\end{frame}

\begin{frame}{Rust principles}
\framesubtitle{Option matching shortcuts}
\large
Some of methods on Option:
\vspace{0.5cm}
\inputminted[fontsize=\Large]{rust}{code/option6.rs}
\end{frame}

\begin{frame}{Rust principles}
\framesubtitle{Option matching shortcuts}
\large
Option is also useful since you can basically take\\
away the value, leaving None in its place:\\
\vspace{0.5cm}
\inputminted[fontsize=\Large]{rust}{code/option7.rs}
\end{frame}

\framepic{graphics/2.jpg}{
 \textcolor{ucuwhite}{Error Handling}
 \vskip 0.5cm
 }

\section{Error Handling}

\begin{frame}{Error handling methods}
	\framesubtitle{Panic}
\large
If you can't recover from an error, just \textcolor{ucuyellow}{panic!}\\
(not irl though)
\vspace{0.9cm}
\inputminted[fontsize=\Large]{rust}{code/error1.rs}
\normalsize
\end{frame}

\begin{frame}{Error handling methods}
\framesubtitle{Working with the result}

\normalsize
If you can recover from an error, use an algebraic\\
type \textcolor{ucuyellow}{Result<T, E>}, which can either be an\\
\textcolor{ucuyellow}{Ok(value of type T)} or 
\textcolor{ucuyellow}{Err(value of type E)}:
\vspace{0.2cm}
\inputminted[fontsize=\Large]{rust}{code/error2.rs}
\vspace{0.55cm}
\end{frame}

\begin{frame}{Error handling methods}
\framesubtitle{Working with the result}
\large
Once again, you can't miss an error this way,\\
you always have to expect it!
\vspace{0.2cm}
\inputminted[fontsize=\Large]{rust}{code/error3.rs}
\vspace{0.55cm}
\end{frame}

\begin{frame}{Error handling methods}
\framesubtitle{Shorthands and syntactic sugar}
\normalsize
\inputminted[fontsize=\Large]{rust}{code/error4.rs}
\end{frame}

\begin{frame}{Error handling methods}
\framesubtitle{Question mark operator}
\inputminted[fontsize=\large]{rust}{code/error5.rs}
\vspace{0.5cm}
\end{frame}

\begin{frame}{Error handling methods}
\framesubtitle{Question mark operator}
\inputminted[fontsize=\large]{rust}{code/error6.rs}
\end{frame}

\framepic{graphics/2.jpg}{
 \textcolor{ucuwhite}{Practice - Linked list}
 \vskip 0.5cm
}

\section{Practice - Linked list}

\begin{frame}{Practice}
\large
Let's implement a basic LinkedList which is going to\\
hold \textcolor{ucuyellow}{u32}s!
\vspace{0.2cm}

It's going to be stack-based (LIFO), so we'd have\\
constant-time insertion and deletion.
\vspace{0.2cm}

Fair Warning: This is going to require some change of thinking!
\end{frame}

\begin{frame}{Practice}
\framesubtitle{Node and heap}
\large
The most basic C/C++ implementation of a node\\
consists of a value and a pointer to a chunk of\\
heap memory with the next node or None.
\vspace{0.2cm}
\inputminted[fontsize=\large]{rust}{code/list1.rs}
\end{frame}

\begin{frame}{Practice}
\framesubtitle{Node and heap}
\large
The most basic C/C++ implementation of a node\\
consists of a value and a pointer to a chunk of\\
heap memory with the next node or None.\\
\vspace{0.2cm}
\textcolor{ucuyellow}{None????? Are you crazy, this is Rust!}
\vspace{0.2cm}
\inputminted[fontsize=\large]{rust}{code/list2.rs}
\end{frame}

\begin{frame}{Practice}
\framesubtitle{Linked list}
\inputminted[fontsize=\normalsize]{rust}{code/list3.rs}
\vspace{0.5cm}
\inputminted[fontsize=\normalsize]{rust}{code/list4.rs}
\vspace{0.5cm}
\end{frame}

\begin{frame}{Practice}
\framesubtitle{Some more functions}
\inputminted[fontsize=\large]{rust}{code/list5.rs}
\end{frame}

\begin{frame}{Practice}
	\framesubtitle{Push and ownership}
\inputminted[fontsize=\normalsize]{rust}{code/list6.rs}
\vspace{0.5cm}
\inputminted[fontsize=\normalsize]{rust}{code/list7.rs}
\end{frame}

\begin{frame}{Practice}
	\framesubtitle{Pop}
\inputminted[fontsize=\large]{rust}{code/list8.rs}
\end{frame}

\begin{frame}{Practice}
	\framesubtitle{Display}
\inputminted[fontsize=\normalsize]{rust}{code/list9.rs}
\end{frame}

\begin{frame}{Practice}
	\framesubtitle{Modules}	
	Let's imagine we have to split Node and LinkedList implementations\\
	into different files. Rust's module system is a little weird so this\\
	little example will help us learn its basics.\\

	\vspace{0.3cm}
	This should be our Node file:
	\vspace{0.3cm}
	\inputminted[fontsize=\normalsize]{rust}{code/list10.rs}
\end{frame}

\begin{frame}{Practice}
	\framesubtitle{Modules}	
	\large
	A few Rust terms:
	\begin{enumerate}
		\item \textbf{Packages}: A Cargo feature that lets
			you build, test, and share multiple crates
		\item \textbf{Crates}: A tree of modules that 
			produces either a library or an executable
		\item \textbf{Modules, pub and use}: Let you control the organization, scope, and privacy of paths
		\item \textbf{Paths}: A way of naming an item, such as a struct, function, or module
	\end{enumerate}
\end{frame}

\begin{frame}{Practice}
	\framesubtitle{Modules}
	\large
	You can use relative and absolute paths to\\
	specify the item you are looking for:\\
	\vspace{0.1cm}
	\inputminted[fontsize=\normalsize]{rust}{code/modules.rs}
\end{frame}

\begin{frame}{Practice}
	\framesubtitle{Modules}
	\large
	And this is the beginning of our LinkedList file:
	\vspace{0.3cm}
	\inputminted[fontsize=\large]{rust}{code/list11.rs}
\end{frame}

\begin{frame}{Practice}
	\framesubtitle{Tests}
	\large
	Rust's ecosystem allows for a quick and easy test\\
	deployment, integrated with all the usual tooling.\\
	Just add this to your linked list source file:\\
	\vspace{0.2cm}
	\inputminted[fontsize=\normalsize]{rust}{code/list12.rs}
	\vspace{0.3cm}
\end{frame}

\begin{frame}{Practice}
	\framesubtitle{Tests}
	\large
	Tests are also super easy to run! Just launch:\\
\mint{bash}{$ cargo test}
	\vspace{0.3cm}
	You can also shorten it to just: 
\mint{bash}{$ cargo t}
	\vspace{0.3cm}
Or launch only the tests you want\\
by specifying their function names: 
\mint{bash}{$ cargo test basic_test}
	
\end{frame}

\framepic{graphics/2.jpg}{
	\textcolor{ucuwhite}{Declarative Macros}
 \vskip 0.5cm
}

\section{Declarative Macros}

\begin{frame}{Macros}
	\framesubtitle{What is that?}
	\large
	Macros are, essentially, a way to metaprogram in Rust.\\
	\vspace{0.2cm}
	Metaprogramming is a way to write code that... writes code.\\
	By transferring the workload of this code from runtime\\
	to compile time we are able to achieve a lot of new\\
	otherwise impossible things.
	
	\vspace{0.5cm}
	\inputminted[fontsize=\large]{rust}{code/macros1.rs}
\end{frame}

\begin{frame}{Macros}
	\framesubtitle{What is that?}
	\large
	There are several kinds of macros in Rust, but\\
	we are only going to talk about declarative\\
	macros, the simplest kind:
	
	\vspace{0.5cm}
	\inputminted[fontsize=\Large]{rust}{code/macros2.rs}
\end{frame}

\begin{frame}{Macros}
	\framesubtitle{And how they work}
	\Large
	Rust's macros are a continuation of the pattern\\
	matching trend we've seen in Rust so far. They\\
	compare an input to a pattern, and substitute it\\
	with code, it's just that the input they take\\
	is Rust code itself.
	
	\vspace{0.2cm}
\end{frame}

\begin{frame}{Macros}
	\framesubtitle{And how they work}
	\inputminted[fontsize=\large]{rust}{code/macros3.rs}
\end{frame}

\framepic{graphics/2.jpg}{
	\textcolor{ucuwhite}{Interesting Rust (community) stuff}
 \vskip 0.5cm
}

\section{Interesting Rust community stuff}

\begin{frame}{Rust editions}
	\Large
	Rust has a six-week release cycle, and there are also\\
	bigger editions released every two-three years, which\\
	might introduce new keywords, implement breaking\\
	changes	and represents a coherent package of stable\\
	changes with updated documentation.\\

\end{frame}

\begin{frame}{Rust editions}
	\large
	So far, Rust has only had 2 editions: Rust 2015 and Rust 2018.\\

	\vspace{0.3cm}
	
	However, Rust 2021 is already in the process of stabilization,\\
	and is supposed to be completely stable by October 21st!\\

	\vspace{0.3cm}
	
	While Rust 2018 introduced a lot of new keywords and breaking\\
	changes (like \textcolor{ucuyellow}{async}/\textcolor{ucuyellow}{await}, \textcolor{ucuyellow}{try}, 
	new path rules etc.), Rust 2021\\
	is more of a "no stress" release.
\end{frame}

\begin{frame}{Rust release channels}
	\large
	All new Rust features go through several stages of\\
	development, discussion and deployment. Rust has\\
	several release channels to show the relative\\
	readiness of new features:\\
	\vspace{0.5cm}
	
	\begin{enumerate}
		\item \textbf{Stable}
		\item \textbf{Beta}
		\item \textbf{Nightly}
	\end{enumerate}
\end{frame}

\begin{frame}{Unsafe and safe Rust}
	\Large
	So far we've only discussed the so called \textcolor{ucuyellow}{safe Rust},\\
	and we are going to continue doing that in the future\\
	lectures, but it's important to know that Rust has an\\
	option to opt out of the compiler's safety checks and\\
	turn your code into \textcolor{ucuyellow}{unsafe} Rust!\end{frame}

\begin{frame}{Unsafe and safe Rust}
	\large
	If you do use unsafe Rust, now you are responsible\\
	for its safety, and have to expose safe APIs to\\
	the outside world.\\
	\vspace{0.5cm}
	\inputminted[fontsize=\large]{rust}{code/unsafe.rs}
\end{frame}

\begin{frame}{Unsafe and safe Rust}
	\Large
	Some of Rust's standard library is implemented with\\
	unsafe code, but it's guaranteed that safe Rust can\\
	never cause undefined behavior! If your code has UB,\\
	it's definitely in the unsafe blocks!\\
\end{frame}

\begin{frame}{Rust community}
	\large
	Rust is being developed as an open-source language, and\\
	shares the values of its open community.\\

	\vspace{0.3cm}
	Every discussion on the future of the language is done\\
	in public with RFCs (Request for Comments), and there\\
	are constantly discussions going on what to work on in\\
	the future!\\

	\vspace{0.3cm}
	Rust Foundation also organizes several working groups\\
	which focus on concrete topics: CLI Rust, Async Rust,\\
	GameDev Rust etc. Currently there are talks about\\
	founding a Rust Education Working Group.
\end{frame}

\framepic{graphics/2.jpg}{
 \textcolor{ucuwhite}{Thank you!}
 \vskip 0.5cm
}

\end{document}
