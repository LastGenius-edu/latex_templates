% Skip over this boring header down a bit
%%%%%%%%%%%%%%%%%%%%%%%%%%%%%%%%%%%%%%%%%%%%%%%%
\documentclass[usenames,twocolumn,dvipsnames,10pt,a4wide]{article} 
\usepackage[a4paper, total={8in, 10in}]{geometry}
% Add option 'aspectratio=169' for 16:9 widescreen 
% Add option  'handout' to ignore animations
% If you have a smaller amount of text, feel free to also try '11pt'! / Jesper

\usepackage[utf8]{inputenc}
\usepackage{verbatim}
\usepackage{minted}
\usemintedstyle{vs}
\usepackage{graphicx}
\usepackage{wrapfig}
\usepackage[document]{ragged2e}

\usepackage[shortlabels]{enumitem}

%%% Bibliography
\usepackage[style=authoryear,backend=biber]{biblatex}
\addbibresource{bibliography.bib}

\DeclareNameAlias{author}{given-family}

%%% Suppress biblatex annoying warning
\usepackage{silence}
\WarningFilter{biblatex}{Patching footnotes failed}

%%% Some useful commands
% pdf-friendly newline in links
\newcommand{\pdfnewline}{\texorpdfstring{\newline}{ }} 
% Fill the vertical space in a slide (to put text at the bottom)
\newcommand{\framefill}{\vskip0pt plus 1filll}

%%%%%%%%%%%%%%%%%%%%%%%%%%%%%%%%%%%%%%%%%%%%%%%%%%%%%%%%%%%%%%%%%%%%%%%%%%%%%%%%%%%%%
\title{Rust \#2: Ownership, Structs and OO}

\begin{document}

\maketitle

\section{Rust basics}
\subsection{Rust primitive types} 
Integer types
\begin{table}[]
\begin{tabular}{lll}
Length  & Signed & Unsigned \\
8-bit   & i8     & u8       \\
16-bit  & i16    & u16      \\
32-bit  & i32    & u32      \\
64-bit  & i64    & u64      \\
128-bit & i128   & u128     \\
arch    & isize  & usize    \\
\end{tabular}
\end{table}

There are also two floating point types:
f64.
And char types.
Characters in Rust are Unicode chars, and can take up to 4 bytes.


\subsection{Rust compound types}
\subsubsection{Tuples}
Tuple groups together a number of values with different
types into one compound type. Tuples have a fixed length.

\inputminted[fontsize=\normalsize]{rust}{code/tuple.rs}



\subsection{Rust compound types}
\subsubsection{Arrays}
Arrays are a collection of elements of the same type,
with a fixed length, allocated on the stack.

\inputminted[fontsize=\normalsize]{rust}{code/array.rs}



\subsection{Rust compound types}
\subsubsection{Slices}
Slices are dynamically-sized "windows", "slices" into a
collection of elements. Slices let the code that handles
them not care whether it's currently working with an
array slice, with a vector slice or something else.

\inputminted[fontsize=\normalsize]{rust}{code/slice.rs}



\subsection{Rust compound types}
\subsubsection{String slices}
String slices are similar, but instead function as windows
into strings. Making your function take a str instead of 
String is preferrable, since the latter can be
downreferenced to slices.

\inputminted[fontsize=\normalsize]{rust}{code/str.rs}



\subsection{Functions}
Rust is primarily an expression language.

Essentially: Expressions evaluate to a value, and
return that value. Statements do not.

\inputminted[fontsize=\normalsize]{rust}{code/expression1.rs}


\subsection{Rust principles}
\subsubsection{Expressions and statements}
\normalsize
Function bodies are made up of a series of statements, 
optionally ending in an expression.


Expressions do not include ending semicolons.


If you add a semicolon to the end of an expression, you turn 
it into a statement, which will then not return a value.
If a function ends in an expression, it returns the value of
that expression.

\inputminted[fontsize=\normalsize]{rust}{code/expression2.rs}



\subsection{Rust principles}
\subsubsection{Common expression usage}
Scopes can return values:
(Rust returns ( ) if nothing is returned, it's like None)
\inputminted[fontsize=\normalsize]{rust}{code/expression3.rs}

if
is also an expression:
\inputminted[fontsize=\normalsize]{rust}{code/expression4.rs}

\normalsize
We can return values from a lot of expressions in Rust 
(match, for example)


\subsection{Rust principles}
\subsubsection{Algebraic data types and match expressions}
Rust uses an interesting concept of algebraic 
data types, which can hold a few types of values. 
An example of this is std::Option:

\inputminted[fontsize=\normalsize]{rust}{code/option1.rs}



\subsection{Rust principles}
\subsubsection{Algebraic data types and match expressions}
\inputminted[fontsize=\normalsize]{rust}{code/option1.rs}

\normalsize
Rust's enums are different from C's since they can hold additional
values inside of them, and Rust's system makes it so you can't
treat one of the values as if it was a different value.


An Option<T>
contains either a Some(value of type T)
or None.
(These are called enum's variants)

Thus, an Option<f64>
is either a Some(f64)
or None.

\subsection{Rust principles}
\subsubsection{Algebraic data types and match expressions}
Let's see how these types can be used:

\inputminted[fontsize=\normalsize]{rust}{code/option2.rs}


\subsection{Rust principles}
\subsubsection{Algebraic data types and match expressions}
Rust forces us to consider all the possible values
of algebraic data types:

\inputminted[fontsize=\normalsize]{rust}{code/option3.rs}

You can never miss an error or have an unexpected value this way!

\subsection{Rust principles}
\subsubsection{Option matching shortcuts}
We can use if let if we only care about the successful\\
case (think of it as a pattern we match against). You\\
can also add an else clause, of course:

\inputminted[fontsize=\normalsize]{rust}{code/option4.rs}

\subsection{Rust principles}
\subsubsection{Option matching shortcuts}
This destructuring can also be used to loop over things,\\
(Rust iterators produce Options, for example):\\

\inputminted[fontsize=\normalsize]{rust}{code/option5.rs}

\subsection{Rust principles}
\subsubsection{Option matching shortcuts}
Some of methods on Option:
\inputminted[fontsize=\normalsize]{rust}{code/option6.rs}

\subsection{Rust principles}
\subsubsection{Option matching shortcuts}
Option is also useful since you can basically take\\
away the value, leaving None in its place:\\
\inputminted[fontsize=\normalsize]{rust}{code/option7.rs}


\section{Error Handling}

\subsection{Error handling methods}
	\subsubsection{Panic}
If you can't recover from an error, just panic!
(not irl though)

\inputminted[fontsize=\normalsize]{rust}{code/error1.rs}
\normalsize


\subsection{Error handling methods}
\subsubsection{Working with the result}

\normalsize
If you can recover from an error, use an algebraic
type Result<T, E>, which can either be an
Ok(value of type T) or 
Err(value of type E):

\inputminted[fontsize=\normalsize]{rust}{code/error2.rs}



\subsection{Error handling methods}
\subsubsection{Working with the result}
Once again, you can't miss an error this way,
you always have to expect it!

\inputminted[fontsize=\normalsize]{rust}{code/error3.rs}



\subsection{Error handling methods}
\subsubsection{Shorthands and syntactic sugar}
\normalsize
\inputminted[fontsize=\normalsize]{rust}{code/error4.rs}


\subsection{Error handling methods}
\subsubsection{Question mark operator}
\inputminted[fontsize=\normalsize]{rust}{code/error5.rs}



\subsection{Error handling methods}
\subsubsection{Question mark operator}
\inputminted[fontsize=\normalsize]{rust}{code/error6.rs}



\section{Practice - Linked list}

\subsection{Practice}
Let's implement a basic LinkedList which is going to
hold u32s!


It's going to be stack-based (LIFO), so we'd have
constant-time insertion and deletion.


Fair Warning: This is going to require some change of thinking!


\subsection{Practice}
\subsubsection{Node and heap}
The most basic C/C++ implementation of a node
consists of a value and a pointer to a chunk of
heap memory with the next node or None.

\inputminted[fontsize=\normalsize]{rust}{code/list1.rs}


\subsection{Practice}
\subsubsection{Node and heap}
The most basic C/C++ implementation of a node
consists of a value and a pointer to a chunk of
heap memory with the next node or None.

None????? Are you crazy, this is Rust!

\inputminted[fontsize=\normalsize]{rust}{code/list2.rs}


\subsection{Practice}
\subsubsection{Linked list}
\inputminted[fontsize=\normalsize]{rust}{code/list3.rs}

\inputminted[fontsize=\normalsize]{rust}{code/list4.rs}



\subsection{Practice}
\subsubsection{Some more functions}
\inputminted[fontsize=\normalsize]{rust}{code/list5.rs}


\subsection{Practice}
	\subsubsection{Push and ownership}
\inputminted[fontsize=\normalsize]{rust}{code/list6.rs}

\inputminted[fontsize=\normalsize]{rust}{code/list7.rs}


\subsection{Practice}
	\subsubsection{Pop}
\inputminted[fontsize=\normalsize]{rust}{code/list8.rs}


\subsection{Practice}
	\subsubsection{Display}
\inputminted[fontsize=\normalsize]{rust}{code/list9.rs}


\subsection{Practice}
	\subsubsection{Modules}	
	Let's imagine we have to split Node and LinkedList implementations
	into different files. Rust's module system is a little weird so this
	little example will help us learn its basics.

	
	This should be our Node file:
	
	\inputminted[fontsize=\normalsize]{rust}{code/list10.rs}


\subsection{Practice}
	\subsubsection{Modules}	
	A few Rust terms:
	\begin{enumerate}
		\item \textbf{Packages}: A Cargo feature that lets
			you build, test, and share multiple crates
		\item \textbf{Crates}: A tree of modules that 
			produces either a library or an executable
		\item \textbf{Modules, pub and use}: Let you control the organization, scope, and privacy of paths
		\item \textbf{Paths}: A way of naming an item, such as a struct, function, or module
	\end{enumerate}


\subsection{Practice}
	\subsubsection{Modules}
	You can use relative and absolute paths to
	specify the item you are looking for:
	
	\inputminted[fontsize=\normalsize]{rust}{code/modules.rs}


\subsection{Practice}
	\subsubsection{Modules}
	And this is the beginning of our LinkedList file:
	
	\inputminted[fontsize=\normalsize]{rust}{code/list11.rs}


\subsection{Practice}
	\subsubsection{Tests}
	Rust's ecosystem allows for a quick and easy test
	deployment, integrated with all the usual tooling.
	Just add this to your linked list source file:
	
	\inputminted[fontsize=\normalsize]{rust}{code/list12.rs}
	


\subsection{Practice}
	\subsubsection{Tests}
	Tests are also super easy to run! Just launch:
\mint{bash}{$ cargo test}
	
	You can also shorten it to just: 
\mint{bash}{$ cargo t}
	
Or launch only the tests you want
by specifying their function names: 
\mint{bash}{$ cargo test basic_test}
	
\section{Declarative Macros}

\subsection{Macros}
	\subsubsection{What is that?}
	Macros are, essentially, a way to metaprogram in Rust.\\
	Metaprogramming is a way to write code that... writes code.\\
	By transferring the workload of this code from runtime\\
	to compile time we are able to achieve a lot of new\\
	otherwise impossible things.
	\inputminted[fontsize=\normalsize]{rust}{code/macros1.rs}

\subsection{Macros}
	\subsubsection{What is that?}
	There are several kinds of macros in Rust, but\\
	we are only going to talk about declarative\\
	macros, the simplest kind:
	\inputminted[fontsize=\normalsize]{rust}{code/macros2.rs}

\subsection{Macros}
	\subsubsection{And how they work}
	Rust's macros are a continuation of the pattern\\
	matching trend we've seen in Rust so far. They\\
	compare an input to a pattern, and substitute it\\
	with code, it's just that the input they take\\
	is Rust code itself.
	

\subsection{Macros}
	\subsubsection{And how they work}
	\inputminted[fontsize=\normalsize]{rust}{code/macros3.rs}


\section{Interesting Rust community stuff}

\subsection{Rust editions}
	Rust has a six-week release cycle, and there are also
	bigger editions released every two-three years, which
	might introduce new keywords, implement breaking
	changes	and represents a coherent package of stable
	changes with updated documentation.



\subsection{Rust editions}
	So far, Rust has only had 2 editions: Rust 2015 and Rust 2018.

	
	
	However, Rust 2021 is already in the process of stabilization,
	and is supposed to be completely stable by October 21st!

	
	
	While Rust 2018 introduced a lot of new keywords and breaking
	changes (like try, 
	new path rules etc.), Rust 2021
	is more of a "no stress" release.


\subsection{Rust community}
	Rust is being developed as an open-source language, and
	shares the values of its open community.

	
	Every discussion on the future of the language is done
	in public with RFCs (Request for Comments), and there
	are constantly discussions going on what to work on in
	the future!

	
	Rust Foundation also organizes several working groups
	which focus on concrete topics: CLI Rust, Async Rust,
	GameDev Rust etc. Currently there are talks about
	founding a Rust Education Working Group.



\end{document}
